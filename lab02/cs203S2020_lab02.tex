\documentclass[11pt]{article}

% NOTE: The "Edit" sections are changed for each assignment

% Edit these commands for each assignment

\newcommand{\assignmentduedate}{January 28}
\newcommand{\assignmentassignedate}{January 21}
\newcommand{\assignmentnumber}{Two}

\newcommand{\labyear}{2020}
\newcommand{\labdueday}{Tuesday}
\newcommand{\labassignday}{Tuesday}
\newcommand{\labtime}{2:30 pm}

\newcommand{\assigneddate}{Assigned: \labassignday, \assignmentassignedate, \labyear{} at \labtime{}}
\newcommand{\duedate}{Due: \labdueday, \assignmentduedate, \labyear{} at \labtime{}}

% NOTE: there was no source code for this assignment
% Instead, the focus was on two Markdown files

\newcommand{\assessment}{\lstinline{assessment.md}}
\newcommand{\conduct}{\lstinline{conduct.md}}
\newcommand{\project}{\lstinline{project.md}}

% Edit these commands to give the name to the main program

\newcommand{\mainprogram}{\lstinline{compute_tf_monolith.py}}
\newcommand{\mainprogramsource}{\lstinline{src/termfrequency/compute_tf_monolith.py}}

% Edit this commands to describe key deliverables

\newcommand{\reflection}{\lstinline{reflection.md}}

% Commands to describe key development tasks

% --> Running gatorgrader.sh
\newcommand{\gatorgraderstart}{\command{gradle grade}}
\newcommand{\gatorgradercheck}{\command{gradle grade}}

% --> Compiling and running program with gradle
\newcommand{\gradlebuild}{\command{gradle build}}
\newcommand{\gradlerun}{\command{gradle run}}

% Commands to describe key git tasks

% NOTE: Could be improved, problems due to nesting

\newcommand{\gitcommitfile}[1]{\command{git commit #1}}
\newcommand{\gitaddfile}[1]{\command{git add #1}}

\newcommand{\gitadd}{\command{git add}}
\newcommand{\gitcommit}{\command{git commit}}
\newcommand{\gitpush}{\command{git push}}
\newcommand{\gitpull}{\command{git pull}}

\newcommand{\gitcommitmainprogram}{\command{git commit src/termfrequency/compute_tf_monolith.py -m "Descriptive commit message"}}

% Commands for the textbooks, since there are so many

\newcommand{\cooperative}{{\em Cooperative Software Design\/}}
\newcommand{\philosophy}{{\em Philosophy of Software Design\/}}
\newcommand{\thinkpython}{{\em Think Python\/}}
\newcommand{\programmingstyle}{{\em Exercises in Programming Style\/}}
\newcommand{\pytest}{{\em Python Testing with Pytest\/}}

% Use this when displaying a new command

\newcommand{\command}[1]{``\lstinline{#1}''}
\newcommand{\program}[1]{\lstinline{#1}}
\newcommand{\url}[1]{\lstinline{#1}}
\newcommand{\channel}[1]{\lstinline{#1}}
\newcommand{\option}[1]{``{#1}''}
\newcommand{\step}[1]{``{#1}''}

\usepackage{pifont}
\newcommand{\checkmark}{\ding{51}}
\newcommand{\naughtmark}{\ding{55}}

\usepackage{listings}
\lstset{
  basicstyle=\small\ttfamily,
  columns=flexible,
  breaklines=true
}

\usepackage{fancyhdr}

\usepackage[margin=1in]{geometry}
\usepackage{fancyhdr}

\pagestyle{fancy}

\fancyhf{}
\rhead{Computer Science 203}
\lhead{Laboratory Assignment \assignmentnumber{}}
\rfoot{Page \thepage}
\lfoot{\duedate}

\usepackage{titlesec}
\titlespacing\section{0pt}{6pt plus 4pt minus 2pt}{4pt plus 2pt minus 2pt}

\newcommand{\labtitle}[1]
{
  \begin{center}
    \begin{center}
      \bf
      CMPSC 203\\Software Engineering\\
      Spring 2020\\
      \medskip
    \end{center}
    \bf
    #1
  \end{center}
}

\begin{document}

\thispagestyle{empty}

\labtitle{Laboratory \assignmentnumber{} \\ \assigneddate{} \\ \duedate{}}

\section*{Objectives}

To continue to practice the use of GitHub and the GitHub Flow model when
preparing a project management plan that you will apply during the completion of
the two long-term software projects.
%
In preparation for the completion of a team-based software project, this
assignment invites you to inventory the skills of your team members and then
make plans about how the team will, for instance, identify software requirements
and organize your time during the laboratory and class sessions.
%
You will also complete a reflection in a GitHub repository created by GitHub
Classroom.

\section*{Suggestions for Success}

\begin{itemize}
  \setlength{\itemsep}{0pt}

\item {\bf Follow each step carefully}. Slowly read each sentence in the
  assignment sheet, making sure that you precisely follow each instruction. Take
  notes about each step that you attempt, recording your questions and ideas and
  the challenges that you faced. If you are stuck, then please tell a technical
  leader or instructor what assignment step you recently completed.

\item {\bf Regularly ask and answer questions}. Please log into Slack at the
  start of a laboratory or practical session and then join the appropriate
  channel. If you have a question about one of the steps in an assignment, then
  you can post it to the designated channel. Or, you can ask a student sitting
  next to you or talk with a technical leader or the course instructor.

\item {\bf Store your files in GitHub}. As in previous laboratory assignments,
  you will be responsible for storing all of your files (e.g., Java source code
  and Markdown-based writing) in a Git repository using GitHub Classroom. Please
  verify that you have saved your source code in your Git repository by using
  \command{git status} to ensure that everything is updated. You can see if your
  assignment submission meets the established correctness requirements by using
  the provided checking tools on your local computer and by checking the commits
  in GitHub.

\item {\bf Keep all of your files}. Don't delete your programs, output files,
  and written reports after you submit them through GitHub; you will need them
  again when you study for the quizzes and examinations and work on the other
  laboratory, practical, and final project assignments.

\item {\bf Explore teamwork and technologies}. While certain aspects of the
  laboratory assignments will be challenging for you, each part is designed to
  give you the opportunity to learn both fundamental concepts in the field of
  computer science and explore advanced technologies that are commonly employed
  at a wide variety of companies. To explore and develop new ideas, you should
  regularly communicate with your team and/or the student technical leaders.

\item {\bf Hone your technical writing skills}. Computer science assignments
  require to you write technical documentation and descriptions of your
  experiences when completing each task. Take extra care to ensure that your
  writing is interesting and both grammatically and technically correct,
  remembering that computer scientists must effectively communicate and
  collaborate with their team members and the student technical leaders and
  course instructor.

\item {\bf Review the Honor Code on the syllabus}. While you may discuss your
  assignments with others, copying source code or writing is a violation of
  Allegheny College's Honor Code.

\end{itemize}

\section*{Reading Assignment}

% Module One Reading Assignment:

% Cooperative Software Design, Chapters 1 - 3
% Philosophy of Software Design, Chapters 1 - 3
% Think Python, Chapters 1 - 3
% Exercises in Programming Style, Prologue, Preface, Chapters 1 - 4
% Python Testing with Pytest, Preface, Chapters 1 - 2

If you have not done so already, please read all of the relevant ``GitHub
Guides'', available at \url{https://guides.github.com/}, that explain how to use
many of the features that GitHub provides. In particular, please make sure that
you have read guides such as ``Mastering Markdown'' and ``Documenting Your
Projects on GitHub''; each of them will help you to understand how to use both
GitHub and GitHub Classroom.
%
To do well on this assignment you should also read Chapters 1 through 3 in in
both \cooperative{} and \philosophy{}.
%
You are also expected to find and read all of the online resources that you need
to complete this laboratory assignment.
%
Please see the course instructor if you have questions on these reading
assignments.

\section*{Creating a Management Plan for Team-based Software Engineering}

% Accept the assignment

To access the laboratory assignment, you should go into the
\channel{\#announcements} channel in our Slack workspace and find the
announcement that provides a link for it. Copy this link and paste it into your
web browser. Now, you should accept the laboratory assignment and see that
GitHub Classroom created a new GitHub repository for you to access the
assignment's starting materials and to store the completed version of your
assignment. Specifically, to access your new GitHub repository for this
assignment, please click the green ``Accept'' button and then click the link
that is prefaced with the label ``Your assignment has been created here''. If
you accepted the assignment and correctly followed these steps, you should have
created a GitHub repository with a name like
``Allegheny-Computer-Science-203-Spring-2020/computer-science-203-spring-2020-lab-2-gkapfham''.
Unless you provide the instructor with documentation of the extenuating
circumstances that you are facing, not accepting the assignment means that you
will receive a failing grade for it.

% Details of what students must complete

Unlike the practical assignments, this laboratory assignment asks you to work in
an entire-class team to write one document that are stored in a separate, shared
GitHub repository. If you look in the repository that GitHub Classroom created,
then you will find the link to a repository that stores the starting version of
a guide supporting the creation of a project management plan.
%
Specifically, your task for this laboratory assignment is to collaborate with
the members of your class to write a full-featured guide for project management
in the context of team-based software engineering. Although it is subject to
revision, you should plan to follow this project management guide throughout the
completion of the two long-term software projects that you will complete as part
of this course. Your guide should provide answers to project management
questions like these:

\begin{itemize}

  \setlength{\itemsep}{0pt}

\item How will you inventory the skills of team members to determine who is
  suited for \mbox{specific tasks}?

\item How will you divide the entire team into sub-teams assigned to complete
  specific tasks?

\item How will you ensure everyone has the opportunity to master all software
  engineering skills?

\item How will you run the course sessions to ensure that you use them
  to maximum effectiveness?

\item When and how frequently will you give demonstrations of the software that
  you \mbox{are engineering}?

\item How will you adjust the project's schedule if the team falls behind on the
  completion of tasks?

\item How and when will your team conduct blame-less postmortems to resolve
  project mistakes?

\item How will your team elicit, analyze, and document the requirements for the
  software project?

\item What will your team do if it determines that it cannot feasibly implement
  promised features?

\item What is your team's software development and testing process for creating
  new features?

\item What process will the team follow to find, report, and fix defects in the
  software project?

\item How will your team know when it is finished with the implementation of the
  software project?

\end{itemize}

Please note that you should use repository forks or branches and pull requests
to ensure that your work is ultimately added to the repository's master branch.
%
When you follow the GitHub Flow model you should create a separate ``feature
branch'' in your repository that contains changes that you want to make to the
project management guide. Please refer to the sheet for the first laboratory
assignment for more details on how the team should use the GitHub Flow model.

% Could not fit this content, which is a repeat from last assignment

%Using appropriate Git commit messages, you will repeatedly commit to this
%branch until you have finished the feature. Now, you will use GitHub to raise a
%pull request, tagging the course instructor and the student technical leaders
%and asking them review your work. Once you have resolved all of the concerns
%raised by the reviewers of your pull request and it is approved by a majority
%of the student technical leaders, a member of your team should merge it into
%the master branch of the GitHub repository. At this point, you can delete the
%feature branch from the repository and start working on the next task
%associated with completing this project.
%%
%Since you can only use GitHub and GitHub Flow to complete this project, the use
%of tools like Google Docs is prohibited for this laboratory assignment.

\section*{Collaborating with Your Software Engineering Team}

Your team should use GitHub and its features (e.g., issue tracker, pull
requests, commit log, and code review request) to complete all of the tasks
referenced in the previous section.
%
Aiming to manage risk and estimate the effort required for individual team
members to complete this project, you should assign people to teams, roles, and
tasks. While it is acceptable for you to have in-person discussions with your
team members or to talk about the project through Slack, please remember that
all important discussions and decisions must be documented through GitHub.
Finally, as you are working with your team, you should carefully document your
experiences and contributions so that you can share them through writing stored
in the repository created by GitHub Classroom.
%
Whenever possible, you should describe your efforts on this laboratory
assignment according to the following levels: N = None, I = Inadequate, A =
Adequate, G = Good, and E = Excellent. Your evaluation of your own work should
focus on your mastery of both technical and professional skills in software
engineering. You should thoughtfully reflect on your current areas of expertise
and opportunities for improvement. Please share your questions about this task
with the instructor.

Since multiple approaches may support the effective completion of the required
document, this assignment does not dictate team organization or communication
strategies. The students in the course should instead work with each other, the
student technical leaders, and the instructor to identify team roles and
strategies for effective organization and communication.
%
You should make sure that every student in the class makes a contribution to the
shared GitHub repository (i.e., the distinct GitHub repository that stores the
software project management guide).
%
Please make sure that you collaborate with your team members to furnish the
answers to software project management questions like those on the previous
page.
%
By the end of the laboratory session, the team should have identified roles for
each student in the class, ensuring that everyone has assigned tasks that will
both capitalize on their expertise and enhance their software engineering
skills.

Please remember that Travis CI is configured to check the Markdown files in the
repository with \command{mdl}.
%
If your writing meets all the established requirements set by this linting
tools, then you will see a green \checkmark{} in the listing of commits in
GitHub after awhile. If your submission does not meet the requirements, a red
\naughtmark{} will appear instead. The instructor will assign your grade in
adherence to a mastery grading paradigm explained in the \command{README.md}
file of your GitHub repository.

\section*{Summary of the Required Deliverables}

% \noindent Students do not need to submit printed source code or technical
% writing for any assignment in this course.

This assignment invites you to submit, only using GitHub, the following
deliverables. Your work will be graded on a checkmark basis for this laboratory
assignment.
%
Unless you provide the course instructor with documentation of the severe and
extenuating circumstances that you are facing, no late work will be considered
towards your completion grade for this practical assignment.

\begin{enumerate}

\setlength{\itemsep}{-.01in}

\item A properly documented, well-formatted, and correct version of \project{}
  that meets all of the established requirements and was collaboratively
  developed by the entire class.

\item Stored in a Markdown file called \reflection{}, a multiple-paragraph
  response that documents and evaluates the work that you completed for this
  laboratory assignment.

\end{enumerate}

\noindent Please submit the first deliverable in the shared, team-based GitHub
repository and submit the second deliverable in your individual, private
repository that was created by GitHub Classroom.

% As you work on this assignment, please don't forget that your software
% engineering team will use the content in the \project{} file to govern the
% completion of the two long-term software projects.

\end{document}
