\documentclass[11pt]{article}

% NOTE: The "Edit" sections are changed for each assignment

% Edit these commands for each assignment

\newcommand{\assignmentduedate}{January 29}
\newcommand{\assignmentassignedate}{January 24}
\newcommand{\assignmentnumber}{One}

\newcommand{\labyear}{2020}
\newcommand{\labdueday}{Wednesday}
\newcommand{\labassignday}{Friday}
\newcommand{\labtime}{1:30 pm}

\newcommand{\assigneddate}{Assigned: \labassignday, \assignmentassignedate, \labyear{} at \labtime{}}
\newcommand{\duedate}{Due: \labdueday, \assignmentduedate, \labyear{} at \labtime{}}

% Edit these commands to give the name to the main program

\newcommand{\mainprogram}{\lstinline{compute_tf_monolith.py}}
\newcommand{\mainprogramsource}{\lstinline{termfrequency/compute_tf_monolith.py}}

\newcommand{\checkprogram}{\lstinline{check_compute_tf_monolith.py}}
\newcommand{\checkprogramsource}{\lstinline{checks/check_compute_tf_monolith.py}}

% Edit this commands to describe key deliverables

\newcommand{\reflection}{\lstinline{writing/reflection.md}}

% Commands to describe key development tasks

% --> Running gatorgrader.sh
\newcommand{\gatorgraderstart}{\command{gradle grade}}
\newcommand{\gatorgradercheck}{\command{gradle grade}}

% --> Compiling and running program with gradle
\newcommand{\gradlebuild}{\command{gradle build}}
\newcommand{\gradlerun}{\command{gradle run}}

% Commands to describe key git tasks

% NOTE: Could be improved, problems due to nesting

\newcommand{\gitcommitfile}[1]{\command{git commit #1}}
\newcommand{\gitaddfile}[1]{\command{git add #1}}

\newcommand{\gitadd}{\command{git add}}
\newcommand{\gitcommit}{\command{git commit}}
\newcommand{\gitpush}{\command{git push}}
\newcommand{\gitpull}{\command{git pull}}

\newcommand{\gitcommitmainprogram}{\command{git commit termfrequency/compute_tf_monolith.py -m "Descriptive commit message"}}

% Commands for the textbooks, since there are so many

\newcommand{\cooperative}{{\em Cooperative Software Design\/}}
\newcommand{\philosophy}{{\em Philosophy of Software Design\/}}
\newcommand{\thinkpython}{{\em Think Python\/}}
\newcommand{\programmingstyle}{{\em Exercises in Programming Style\/}}
\newcommand{\pytest}{{\em Python Testing with Pytest\/}}

% Use this when displaying a new command

\newcommand{\command}[1]{``\lstinline{#1}''}
\newcommand{\program}[1]{\lstinline{#1}}
\newcommand{\url}[1]{\lstinline{#1}}
\newcommand{\channel}[1]{\lstinline{#1}}
\newcommand{\option}[1]{``{#1}''}
\newcommand{\step}[1]{``{#1}''}

\usepackage{pifont}
\newcommand{\checkmark}{\ding{51}}
\newcommand{\naughtmark}{\ding{55}}

\usepackage{listings}
\lstset{
  basicstyle=\small\ttfamily,
  columns=flexible,
  breaklines=true
}

\usepackage{fancyhdr}

\usepackage[margin=1in]{geometry}
\usepackage{fancyhdr}

\pagestyle{fancy}

\fancyhf{}
\rhead{Computer Science 203}
\lhead{Practical Assignment \assignmentnumber{}}
\rfoot{Page \thepage}
\lfoot{\duedate}

\usepackage{titlesec}
\titlespacing\section{0pt}{6pt plus 4pt minus 2pt}{4pt plus 2pt minus 2pt}

\newcommand{\labtitle}[1]
{
  \begin{center}
    \begin{center}
      \bf
      CMPSC 203\\Software Engineering\\
      Spring 2020\\
      \medskip
    \end{center}
    \bf
    #1
  \end{center}
}

\begin{document}

\thispagestyle{empty}

\labtitle{Practical \assignmentnumber{} \\ \assigneddate{} \\ \duedate{}}

\section*{Objectives}

To practice using GitHub to access the files for a practical assignment.
Additionally, to practice using the your laptop's operating system and software
development programs such as a ``terminal window'' and a ``text editor''. You
will also continue to practice using Slack to support communication with the
technical leaders and the course instructor. Next, you will learn how to
implement and run a Python program, also discovering how to use the Pipenv tool
and the course's automated grading tool to assess your progress towards
correctly completing the project. Finally, you will explore how to perform
automated testing with programs implemented in the monolithic style.

\section*{Suggestions for Success}

\begin{itemize}
  \setlength{\itemsep}{0pt}

\item {\bf Follow each step carefully}. Slowly read each sentence in the
  assignment sheet, making sure that you precisely follow each instruction. Take
  notes about each step that you attempt, recording your questions and ideas and
  the challenges that you faced. If you are stuck, then please tell a technical
  leader or instructor what assignment step you recently completed.

\item {\bf Regularly ask and answer questions}. Please log into Slack at the
  start of a laboratory or practical session and then join the appropriate
  channel. If you have a question about one of the steps in an assignment, then
  you can post it to the designated channel. Or, you can ask a student sitting
  next to you or talk with a technical leader or the course instructor.

\item {\bf Store your files in GitHub}. Starting with this laboratory
  assignment, you will be responsible for storing all of your files (e.g., Java
  source code and Markdown-based writing) in a Git repository using GitHub
  Classroom. Please verify that you have saved your source code in your Git
  repository by using \command{git status} to ensure that everything is
  updated. You can see if your assignment submission meets the established
  correctness requirements by using the provided checking tools on your local
  computer and by checking the commits in GitHub.

\item {\bf Keep all of your files}. Don't delete your programs, output files,
  and written reports after you submit them through GitHub; you will need them
  again when you study for the quizzes and examinations and work on the other
  laboratory, practical, and final project assignments.

\item {\bf Explore teamwork and technologies}. While certain aspects of the
  laboratory assignments will be challenging for you, each part is designed to
  give you the opportunity to learn both fundamental concepts in the field of
  computer science and explore advanced technologies that are commonly employed
  at a wide variety of companies. To explore and develop new ideas, you should
  regularly communicate with your team and/or the student technical leaders.

\item {\bf Hone your technical writing skills}. Computer science assignments
  require to you write technical documentation and descriptions of your
  experiences when completing each task. Take extra care to ensure that your
  writing is interesting and both grammatically and technically correct,
  remembering that computer scientists must effectively communicate and
  collaborate with their team members and the student technical leaders and
  course instructor.

\item {\bf Review the Honor Code on the syllabus}. While you may discuss your
  assignments with others, copying source code or writing is a violation of
  Allegheny College's Honor Code.

\end{itemize}

\section*{Reading Assignment}

If you have not done so already, please read all of the relevant ``GitHub
Guides'', available at \url{https://guides.github.com/}, that explain how to use
many of the features that GitHub provides. In particular, please make sure that
you have read guides such as ``Mastering Markdown'' and ``Documenting Your
Projects on GitHub''; each of them will help you to understand how to use both
GitHub and GitHub Classroom. To do well on this assignment, you should also read
Chapters 1 and 2 in {\em Think Python\/} and the Preface and Chapters 1 through
3 in the {\em Exercises in Programming Style\/}.
%
You are also expected to find and read all of the online resources that
you need to complete this assignment.
%
Please see the course instructor if you have questions on these reading
assignments.

\section*{Debugging and Testing a Monolithic Python Program}

As you are starting this assignment, you first need to complete these tasks to
ensure that your laptop is configured correctly to use GitHub, Docker, Pyenv,
and Pipenv:

\begin{itemize}
  \setlength{\itemsep}{0pt}

  \item Ensure that you have a GitHub account that is connected to your
    Allegheny-provided email.

  \item Create and upload SSH keys to your GitHub account, enabling secure
    access to source code.

  \item Install, configure, and test the Docker Desktop Community
    Edition program.

  \item Install the Pyenv tool and use it to install a recent version of Python.

  \item Install the Pipenv tool and later use it to install a Python
    application's dependencies

\end{itemize}

To access the practical assignment, you should go into the
\channel{\#announcements} channel in our Slack team and find the announcement
that provides a link for it. Copy this link and paste it into your web browser.
Now, you should accept the practical assignment and see that GitHub Classroom
created a new GitHub repository for you to access the assignment's starting
materials and to store the completed version of your assignment. Specifically,
to access your new GitHub repository for this assignment, please click the green
``Accept'' button and then click the link that is prefaced with the label ``Your
assignment has been created here''. If you accepted the assignment and correctly
followed these steps, you should have created a GitHub repository with a name
like
``Allegheny-Computer-Science-203-Spring-2020/computer-science-203-spring-2020-practical-1-gkapfham''.
Unless you provide the instructor with documentation of the extenuating
circumstances that you are facing, not accepting the assignment means that you
automatically receive a failing grade for it. Now, remember that your home base
for this assignment is the directory that contains all of the configuration
files and the \command{termfrequency/} directory. Make sure that you can find
your ``home base'' for this practical assignment! Please see the instructor if
you are stuck on getting started.

Now, study the provided Python source code and the technical documentation to
understand the type of output that your program should produce.
%
At the outset, you should notice that the provided source code does not contain
all of the source code from Chapter 3 of the book. You will need to add in the
appropriate source code and documentation to ensure that the Python program
passes all of the checks and produces the correct output. As you complete this
assignment, please make sure that you understand and document all aspects of the
provided Python programs!

Once you have installed Docker Desktop, you can use the following
\command{docker run} command to start \command{gradle grade} as a containerized
application, using the ``DockaGator'' Docker image available on DockerHub. You
can run the following command to run the \command{gradle grade} on your project:

\begin{verbatim}
docker run --rm --name dockagator \
  -v "$(pwd)":/project \
  -v "$HOME/.dockagator":/root/.local/share \
  gatoreducator/dockagator
\end{verbatim}

The aforementioned command will use \program{"\$(pwd)"} (i.e., the current
directory) as the project directory and \program{"\$HOME/.dockagator"} as the
cached GatorGrader directory. Please note that both of these directories must
exist, although only the project directory must contain some content. Generally,
the project directory should contain the source code and technical writing for
this assignment, as provided to you through GitHub during the completion of a
previous step. Additionally, the cache directory should not contain anything
other than directories and programs created by DockaGator, thus ensuring that
they are not otherwise overwritten during the completion of the assignment. To
ensure that the previous command will work correctly, you should create the
cache directory by running the command \command{mkdir \$HOME/.dockagator}; you
will only need to do this once. If the above \command{docker run} command does
not work correctly on the Windows operating system, then you may need to instead
run the following command to work around limitations in the terminal window:

\begin{verbatim}
docker run --rm --name dockagator \
  -v "%cd%:/project" \
  -v "%HomeDrive%%HomePath%/.dockagator:/root/.local/share" \
  gatoreducator/dockagator
\end{verbatim}

To enter into an ``interactive terminal'' in the Docker container, you can add
\command{-it} before \command{--rm} and \command{/bin/bash} to the end of the
previous \command{docker run} command that you typed. If you are stuck on this
step, check your GitHub repository's documentation for more details!
%
In addition to being able to run the grading command in a Docker container, you
should also be able to run all of the Python programs in your terminal using the
previously installed Pyenv and Pipenv programs. Please note that this means
that, after completing this assignment, you should be able to run both of the
following commands inside of either a Docker container or your terminal window.

\begin{itemize}

  \item \program{pipenv run python termfrequency/compute_tf_monolith.py inputs/input.txt}
  \item \program{pipenv run python termfrequency/compute_tf_monolith.py inputs/pride-and-prejudice.txt}

\end{itemize}

\vspace*{-.2in}

For instance, if you run the first of the above commands, then it will produce
the following correct output in your terminal window. If you are not sure what
this output means or how the program created it, then review the relevant
content in the \programmingstyle{} book.

\vspace*{-.1in}

\begin{verbatim}
  live  -  2
  mostly  -  2
  white  -  1
  tigers  -  1
  india  -  1
  wild  -  1
  lions  -  1
  africa  -  1
\end{verbatim}

\section*{Checking the Correctness of Your Program and Writing}

Next, you need to implement and explain some approach to automatically testing
this monolithic Python program! Your approach should be executable at the
command-line and always ensure that the program produces the expected output for
both the small input file. To start this phase of the assignment, please review
the source code in the \checkprogramsource{} file. You will notice that this is
a Python program that calls the \mainprogram{} to see if it produces the
expected output. Since this program is not yet completed, you should carefully
review all of the \command{TODO} markers and add each of the requested source
code statements.
%
For instance, the following conditional logic will check whether or not the
\mainprogram{} produces the first line in the expected output. Along with
resolving every remaining \command{TODO} markers in the source code, you will
need to add all of the additional checks for the other lines in the program's
output.

\begin{verbatim}
  if "live  -  2" not in decoded_stdout:
      print(" - `live  -  2' is not detected")
      exit_code = 1
\end{verbatim}

Before you move onto the next steps of this assignment, please make sure that
you understand how the following Python source code lines check whether or not
the \mainprogramsource{} is working correctly. You should also take time to
reflect on how the use of the monolithic program style limits a software
engineer's ability to effectively test \mainprogram. Now that you have finished
implementing the \checkprogramsource{} you should also think about the
limitations of this approach to testing a Python program. Finally, can you think
of a way in which you code refactor the source code of \mainprogram{} so that it
would be easier to test it? What are the trade-offs associated with this
alternative approach to implementing \mainprogram? Which approach do you prefer?
Why do you prefer this approach?

\begin{verbatim}
  # assume that all of the checks run correctly and prove otherwise
  exit_code = 0

  # define the command that will call the check_compute_tf_monolith.py
  command = "pipenv run python termfrequency/compute_tf_monolith.py inputs/input.txt"

  # tokenize this command so that subprocess can accept each of its parts
  tokenized_command = shlex.split(command)
  print("Tokenized command to execute: " + str(tokenized_command))
  print()

  # run the tokenized command and display the output as a byte string
  result = subprocess.run(tokenized_command, stdout=subprocess.PIPE, check=True)
  print("Output of executed command: " + str(result.stdout))
  print()

  # decode the byte string using UTF-8
  decoded_stdout = str(result.stdout.decode("utf-8"))
\end{verbatim}

After completing the source code in \checkprogramsource{} you can run this
program by typing \command{pipenv run python
checks/check_compute_tf_monolith.py} in your terminal window.
%
Along with running this automated checking tool, you can use GatorGrader to
check other characteristics of your project. To automatically check your Python
source code you can get started with the use of the GatorGrader tool, typing the
command \gatorgraderstart{} in your Docker container.
%
Once your program is running correctly, fulfilling at least some of the
assignment's requirements, you should transfer your files to GitHub using the
\gitcommit{} and \gitpush{} commands. For example, if you want to signal that
the \mainprogramsource{} file has been changed and is ready for transfer to
GitHub you would first type \gitcommitmainprogram{} in your terminal, followed
by typing \gitpush{} and checking to see that the transfer to GitHub is
successful. If you notice that transferring your Python source code to GitHub
did not work correctly, then please try to determine why, asking a technical
leader or the course instructor for assistance, if necessary.

% If you do have mistakes in your assignment, then you will need to review
% GatorGrader's output, find the mistake, and try to fix it.

After the course instructor enables \step{continuous integration} with a system
called Travis CI, when you use the \gitpush{} command to transfer your source
code to your GitHub repository, Travis CI will initialize a \step{build} of your
assignment, checking to see if it meets all of the requirements. If both your
source code and writing meet all of the established requirements, then you will
see a green \checkmark{} in the listing of commits in GitHub after awhile. If
your submission does not meet the requirements, a red \naughtmark{} will appear
instead. The instructor will reduce a student's grade for this assignment if the
red \naughtmark{} appears on the last commit in GitHub immediately before the
assignment's due date. Yet, if the green \checkmark{} appears on the last commit
in your GitHub repository, then you satisfied all of the main checks and you
will earn full credit for one part of your grade. Unless you provide the course
instructor with documentation of the severe and extenuating circumstances that
you are facing, no late work will be considered towards your completion grade
for this assignment.

\section*{Summary of the Required Deliverables}

\noindent Students do not need to submit printed source code or technical
writing for any assignment in this course. Instead, this assignment invites you
to submit, using GitHub, the following deliverables.

% Because this is a practical assignment, you are not required to complete any
% technical writing.

\begin{enumerate}

\setlength{\itemsep}{0in}

\item A properly documented, well-formatted, and correct version of
  \mainprogramsource{} that both meets all of the established requirements and
  produces the desired output.

\item A properly documented, well-formatted, and correct version of
  \checkprogramsource{} that both meets all of the established requirements and
  produces the desired output.

\item Stored in a Markdown file called \reflection{}, a detailed response to all
  of the stated prompts and a code block showing your program's output on the
  small input file.

\end{enumerate}

\section*{Evaluation of Your Practical Assignment}

Using a report that the instructor shares with you through the commit log in
GitHub, you will privately received a grade on this assignment and feedback on
your submitted deliverables. Your grade for the assignment will be a function of
the whether or not it was submitted in a timely fashion and if your program
received a green \checkmark{} indicating that it met all of the requirements.
Other factors will also influence your final grade on the assignment. In
addition to studying the efficiency and effectiveness and documentation of your
Python source code, the instructor will also evaluate the correctness of your
technical writing. If your submission receives a red \naughtmark{}, the
instructor will reduce your grade for the assignment. Please remember to read
your GitHub repository's \program{README.md} file for a description of the four
grades that you will receive for this practical assignment. Students should
follow the instructions in the \program{README.md} file to ask the instructor to
review their assignment and to provide you with constructive feedback before you
submit it for the final assessment.

\section*{Adhering to the Honor Code}

In adherence to the Honor Code, students should complete this assignment on an
individual basis. While it is appropriate for students in this class to have
high-level conversations about the assignment, it is necessary to distinguish
carefully between the student who discusses the principles underlying a problem
with others and the student who produces assignments that are identical to, or
merely variations on, someone else's work. Deliverables (e.g., Python source
code) that are nearly identical to the work of others will be taken as evidence
of violating the \mbox{Honor Code}. Please see the course instructor during
office hours if you have questions about the Honor Code policy.

\end{document}
