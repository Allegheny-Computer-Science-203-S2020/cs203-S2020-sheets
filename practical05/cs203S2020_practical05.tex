\documentclass[11pt]{article}

% NOTE: The "Edit" sections are changed for each assignment

% Edit these commands for each assignment

\newcommand{\assignmentduedate}{February 26}
\newcommand{\assignmentassignedate}{February 21}
\newcommand{\assignmentnumber}{Five}

\newcommand{\labyear}{2020}
\newcommand{\labdueday}{Wednesday}
\newcommand{\labassignday}{Friday}
\newcommand{\labtime}{1:30 pm}

\newcommand{\assigneddate}{Assigned: \labassignday, \assignmentassignedate, \labyear{} at \labtime{}}
\newcommand{\duedate}{Due: \labdueday, \assignmentduedate, \labyear{} at \labtime{}}

% Edit these commands to give the name to the main program

\newcommand{\mainprogram}{\lstinline{compute_tf_pipeline.py}}
\newcommand{\mainprogramsource}{\lstinline{termfrequency/compute_tf_pipeline.py}}

\newcommand{\testprogram}{\lstinline{test_compute_tf_pipeline.py}}
\newcommand{\testprogramsource}{\lstinline{tests/test_compute_tf_pipeline.py}}

% Edit this commands to describe key deliverables

\newcommand{\reflection}{\lstinline{writing/reflection.md}}

% Commands to describe key development tasks

% --> Running gatorgrader.sh
\newcommand{\gatorgraderstart}{\command{gradle grade}}
\newcommand{\gatorgradercheck}{\command{gradle grade}}

% --> Compiling and running program with gradle
\newcommand{\gradlebuild}{\command{gradle build}}
\newcommand{\gradlerun}{\command{gradle run}}

% Commands to describe key git tasks

% NOTE: Could be improved, problems due to nesting

\newcommand{\gitcommitfile}[1]{\command{git commit #1}}
\newcommand{\gitaddfile}[1]{\command{git add #1}}

\newcommand{\gitadd}{\command{git add}}
\newcommand{\gitcommit}{\command{git commit}}
\newcommand{\gitpush}{\command{git push}}
\newcommand{\gitpull}{\command{git pull}}

\newcommand{\gitcommitmainprogram}{\command{git commit termfrequency/compute_tf_pipline.py -m "Descriptive commit message"}}
\newcommand{\gitcommittestprogram}{\command{git commit tests/test_compute_tf_pipline.py -m "Descriptive commit message"}}

% Commands for the textbooks, since there are so many

\newcommand{\cooperative}{{\em Cooperative Software Design\/}}
\newcommand{\philosophy}{{\em Philosophy of Software Design\/}}
\newcommand{\thinkpython}{{\em Think Python\/}}
\newcommand{\programmingstyle}{{\em Exercises in Programming Style\/}}
\newcommand{\pytest}{{\em Python Testing with Pytest\/}}

% Use this when displaying a new command

\newcommand{\command}[1]{``\lstinline{#1}''}
\newcommand{\program}[1]{\lstinline{#1}}
\newcommand{\url}[1]{\lstinline{#1}}
\newcommand{\channel}[1]{\lstinline{#1}}
\newcommand{\option}[1]{``{#1}''}
\newcommand{\step}[1]{``{#1}''}

\usepackage{pifont}
\newcommand{\checkmark}{\ding{51}}
\newcommand{\naughtmark}{\ding{55}}

\usepackage{listings}
\lstset{
  basicstyle=\small\ttfamily,
  columns=flexible,
  breaklines=true
}

\usepackage{fancyhdr}

\usepackage[margin=1in]{geometry}
\usepackage{fancyhdr}

\pagestyle{fancy}

\fancyhf{}
\rhead{Computer Science 203}
\lhead{Practical Assignment \assignmentnumber{}}
\rfoot{Page \thepage}
\lfoot{\duedate}

\usepackage{titlesec}
\titlespacing\section{0pt}{6pt plus 4pt minus 2pt}{4pt plus 2pt minus 2pt}

\newcommand{\labtitle}[1]
{
  \begin{center}
    \begin{center}
      \bf
      CMPSC 203\\Software Engineering\\
      Spring 2020\\
      \medskip
    \end{center}
    \bf
    #1
  \end{center}
}

\begin{document}

\thispagestyle{empty}

\labtitle{Practical \assignmentnumber{} \\ \assigneddate{} \\ \duedate{}}

\section*{Objectives}

To practice using GitHub to access the files for a practical assignment.
Additionally, to practice using the Ubuntu operating system and software
development programs such as a ``terminal window'' and a ``testing tool''. You
will also continue to practice using Slack to support communication with the
teaching assistants and the course instructor. You will carry on with practice
in implementing and running a Python program and also using Pipenv and the
course's automated grading tool to assess your progress towards correctly
completing the project. Along with practicing how to use Pytest plugins, you
will refactor the code of a Python program implemented in the pipeline
style and then use parameterized testing to establish a confidence in the
program's correctness.

\section*{Suggestions for Success}

\begin{itemize}
  \setlength{\itemsep}{0pt}

\item {\bf Follow each step carefully}. Slowly read each sentence in the
  assignment sheet, making sure that you precisely follow each instruction. Take
  notes about each step that you attempt, recording your questions and ideas and
  the challenges that you faced. If you are stuck, then please tell a technical
  leader or instructor what assignment step you recently completed.

\item {\bf Regularly ask and answer questions}. Please log into Slack at the
  start of a laboratory or practical session and then join the appropriate
  channel. If you have a question about one of the steps in an assignment, then
  you can post it to the designated channel. Or, you can ask a student sitting
  next to you or talk with a technical leader or the course instructor.

\item {\bf Store your files in GitHub}. Starting with this laboratory
  assignment, you will be responsible for storing all of your files (e.g., Java
  source code and Markdown-based writing) in a Git repository using GitHub
  Classroom. Please verify that you have saved your source code in your Git
  repository by using \command{git status} to ensure that everything is
  updated. You can see if your assignment submission meets the established
  correctness requirements by using the provided checking tools on your local
  computer and by checking the commits in GitHub.

\item {\bf Keep all of your files}. Don't delete your programs, output files,
  and written reports after you submit them through GitHub; you will need them
  again when you study for the quizzes and examinations and work on the other
  laboratory, practical, and final project assignments.

\item {\bf Explore teamwork and technologies}. While certain aspects of the
  laboratory assignments will be challenging for you, each part is designed to
  give you the opportunity to learn both fundamental concepts in the field of
  computer science and explore advanced technologies that are commonly employed
  at a wide variety of companies. To explore and develop new ideas, you should
  regularly communicate with your team and/or the student technical leaders.

\item {\bf Hone your technical writing skills}. Computer science assignments
  require to you write technical documentation and descriptions of your
  experiences when completing each task. Take extra care to ensure that your
  writing is interesting and both grammatically and technically correct,
  remembering that computer scientists must effectively communicate and
  collaborate with their team members and the student technical leaders and
  course instructor.

\item {\bf Review the Honor Code on the syllabus}. While you may discuss your
  assignments with others, copying source code or writing is a violation of
  Allegheny College's Honor Code.

\end{itemize}

\section*{Reading Assignment}

If you have not done so already, please read all of the relevant ``GitHub
Guides'', available at \url{https://guides.github.com/}, that explain how to use
many of the features that GitHub provides. In particular, please make sure that
you have read guides such as ``Mastering Markdown'' and ``Documenting Your
Projects on GitHub''; each of them will help you to understand how to use both
GitHub and GitHub Classroom. To do well on this assignment, you should also read
Chapters 1 through 5 in {\em Think Python\/} and the Preface and Chapters 1
through 5 in the {\em Exercises in Programming Style\/}.
%
You should also read the first two chapters in \pytest, making sure that you
understand parameterized testing.
%
Please make sure that you review the relevant content in \cooperative~and
\philosophy.
%
You are also expected to find and read all of the online resources that you need
to complete this assignment.

\section*{Refactoring and Testing a Pipeline-Style Program}

To access the practical assignment, you should go into the
\channel{\#announcements} channel in our Slack team and find the announcement
that provides a link for it. Copy this link and paste it into your web browser.
Now, you should accept the practical assignment and see that GitHub Classroom
created a new GitHub repository for you to access the assignment's starting
materials and to store the completed version of your assignment. Specifically,
to access your new GitHub repository for this assignment, please click the green
``Accept'' button and then click the link that is prefaced with the label ``Your
assignment has been created here''. If you accepted the assignment and correctly
followed these steps, you should have created a GitHub repository with a name
like
``Allegheny-Computer-Science-203-Spring-2020/computer-science-203-spring-2020-practical-5-gkapfham''.
Unless you provide the instructor with documentation of the extenuating
circumstances that you are facing, not accepting the assignment means that you
automatically receive a failing grade for it.

Study the provided source code and the technical documentation to understand the
type of output that your program produces.
%
Make sure that you notice that this source code is incorrectly formatted and
written in a fashion that uses deprecated features of the Python programming
language.
%
Part of your job for this assignment is to completely refactor the source code
so that it meets all of the well-established standards for Python programs.
%
As you are improving all of the provided code, you should also run the Pytest
test suite that currently contains two tests. Following the strategy of these
tests and the documentation for the functions in the program, add all of the
test cases that you would need to achieve high coverage of the program under
test.
%
You should also write parameterized Pytest test cases that exercise the
functions in this Pipeline-style program with numerous inputs. To implement a
parameterized test case, you must decorate a Python testing function with the
\command{@pytest.mark.parametrize} annotation, as in Figure~\ref{fig:param}. How
do you think that parameterized testing makes it easier to better achieve a
confidence in the correctness of the program under test? What are the
limitations of parameterized testing? Are there programming styles that do not
support the use of parameterized testing?
%
Although you are not required to write all of your tests in a parameterized
fashion, at least some of the new tests that you write must adopt this approach.
%
Please see the instructor if you have questions about either how to refactor the
program's source code to meet the standard or how to write and run parameterized
tests.

\begin{figure}[t]

\begin{verbatim}
@pytest.mark.parametrize(
    "input_string,expected_count",
    [("hello world", 2), ("hello world example", 3), ("", 0), (" ", 0), ("  ", 0)],
)
def test_scan_splits_string_correctly_again(input_string, expected_count):
    """Check that scan function finds the correct number of words in the String."""
    assert len(compute_tf_pipeline.scan(input_string)) == expected_count
\end{verbatim}

\caption{A Parameterized Test Case for the {\tt scan} Function in {\tt compute\_tf\_pipeline}.}\label{fig:param}

\end{figure}

%%
%If you have already installed Pipenv and the project's development dependencies,
%you can run the tests by typing \command{pipenv run pytest}.
%%
%After that, please try to use Pytest plugins like \program{pytest-clarity} and
%\program{pytest-sugar} and observing how they help you to be a better tester.
%%
%Finally, review the ``red flags'' associated with poor software designs and try
%to take steps to refactor the Python program and tests --- while preserving the
%style --- to avoid those concerns, documenting your steps.
%%
%Instead of fully repairing the provided program, the point of this assignment is
%to develop your awareness of red flags and to address those problems that you
%can (e.g., problems with source code documentation) without stepping outside of
%the program style.
%%
%In follow-on assignments you will learn more about how to overcome the
%deficiencies inherent in this style.
%%
%For instance, parameterized unit testing is more feasible if functions accept
%parameters.

\section*{Checking the Correctness of Your Program and Writing}

In addition to using your own automated Pytest-based tests, you can use
GatorGrader to check other characteristics of your project. To automatically
check your Python source code you can get started with the use of the
GatorGrader tool, typing the command \gatorgraderstart{} in your terminal
window.
%
Once your program is running correctly, fulfilling at least some of the
assignment's requirements, you should transfer your files to GitHub using the
\gitcommit{} and \gitpush{} commands. For example, if you want to signal that
the \testprogramsource{} file has been changed and is ready for transfer to
GitHub you would first type \gitcommittestprogram{} in your terminal, followed
by typing \gitpush{} and checking to see that the transfer to GitHub is
successful.

% If you notice that transferring your Python source code to GitHub
% did not work correctly, then please try to determine why, asking a teaching
% assistant or the course instructor for assistance, if necessary.

After the course instructor enables \step{continuous integration} with a system
called Travis CI, when you use the \gitpush{} command to transfer your source
code to your GitHub repository, Travis CI will initialize a \step{build} of your
assignment, checking to see if it meets all of the requirements. If both your
source code and writing meet all of the established requirements, then you will
see a green \checkmark{} in the listing of commits in GitHub after awhile. If
your submission does not meet the requirements, a red \naughtmark{} will appear
instead. The instructor will reduce a student's grade for this assignment if the
red \naughtmark{} appears on the last commit in GitHub immediately before the
assignment's due date. Yet, if the green \checkmark{} appears on the last commit
in your GitHub repository, then you satisfied all of the main checks and you
will earn full credit. Unless you provide the course instructor with
documentation of the severe and extenuating circumstances that you are facing,
no late work will be considered towards your completion grade for this practical
assignment.

\section*{Summary of the Required Deliverables}

\noindent Students do not need to submit printed source code or technical
writing for any assignment in this course. Instead, this assignment invites you
to submit, using GitHub, the following deliverables.

\vspace*{-.25em}

\begin{enumerate}

\setlength{\itemsep}{0in}

\item A properly documented, well-formatted, and correct version of
  \testprogramsource{} that contains at least five passing tests and produces
  the desired diagnostic output. Make sure that some of the tests adhere to the
  parameterized testing approach supported by Pytest.

\item A revised version of \mainprogramsource{} that, when appropriate, contains
  refactored source code that avoids all of the deprecated features of the
  Python programming language and is in adherence with the established standard
  for writing Python source code.

\item Stored in a Markdown file called \reflection{}, a multiple-paragraph
  response, with each paragraph consisting of at least 100 words, for each of
  the stated prompts. This file should feature a fenced code block that gives
  the output from running a test suite.

\end{enumerate}

\vspace*{-1em}

\section*{Evaluation of Your Practical Assignment}

Practical assignments are graded on a completion --- or ``checkmark'' --- basis.
If your GitHub repository has a \checkmark{} for the last commit before the
deadline then you will receive the highest possible grade for the assignment.
However, you will fail the assignment if you do not complete it correctly, as
evidenced by a red \naughtmark{} in your commit listing, by the set deadline for
finishing the project.

% In adherence to the Honor Code, students should complete this assignment on an
% individual basis. While it is appropriate for students in this class to have
% high-level conversations about the assignment, it is necessary to distinguish
% carefully between the student who discusses the principles underlying a problem
% with others and the student who produces assignments that are identical to, or
% merely variations on, someone else's work. Deliverables (e.g., Python source
% code) that are nearly identical to the work of others will be taken as evidence
% of violating the \mbox{Honor Code}. Please see the course instructor during
% office hours if you have questions about the Honor Code policy.

\end{document}
